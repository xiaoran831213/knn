\documentclass[11pt]{article} \usepackage{textcomp,bbding,subfig}
\usepackage{float,amssymb,amsmath,amsfonts,bm}
\usepackage{graphicx,cite} \usepackage[]{natbib} \def\style{apa}
\usepackage[usenames,pdftex,dvips]{color,xcolor}
\usepackage{multirow,tabulary,colortbl,array}
\usepackage[normalem]{ulem}
\usepackage[colorlinks,bookmarksopen,bookmarksnumbered,citecolor=blue,urlcolor=blue]{hyperref}
\usepackage{moreverb,setspace} \usepackage{algpseudocode}
\usepackage{algorithm}
%
% enable numbered boxes, with caption and label.
\makeatletter \newcommand\fs@boxedtop { \fs@boxed
  \def\@fs@mid{\vspace\abovecaptionskip\relax}
  \let\@fs@iftopcapt\iftrue } \makeatother \floatstyle{boxedtop}
\floatname{framedbox}{Box} \newfloat{framedbox}{tbp}{lob}
%
% Text layout
\topmargin -1.5cm \oddsidemargin 0.0cm \evensidemargin 0.0cm
\textwidth 16.5cm \textheight 23.5cm \setlength{\parindent}{0cm}

% Remove brackets from numbering in List of References \makeatletter
% \renewcommand\@biblabel[1]{} \makeatother
\makeatletter \renewcommand{\@biblabel}[1]{\quad#1.}  \makeatother

% xiaoran's definitions
% xtong's tools
% aliasis
\newcommand{\xemp}[1]{{\color{red}{\textbf{#1}}}}
\newcommand{\bs}{\boldsymbol}
\newcommand{\mean}[2]{\left\langle{#1}\right\rangle_{#2}}
\newcommand{\trb}[1]{\textrm{Tr}\left({#1}\right)}
\newcommand{\trs}[1]{\textrm{Tr}\left[{#1}\right]}
\newcommand{\invb}[1]{{\left({#1}\right)^-}}
\newcommand{\invs}[1]{{\left[{#1}\right]^-}}
\newcommand\numberthis{\addtocounter{equation}{1}\tag{\theequation}}
\renewcommand{\eqref}[1]{Eq.\,\ref{#1}}
%
% vectors and matrices
\newcommand{\va}{\mathbf{a}}
\newcommand{\vb}{\mathbf{b}}
\newcommand{\vc}{\mathbf{c}}
\newcommand{\vf}{\mathbf{f}}
\newcommand{\vg}{\mathbf{g}}
\newcommand{\vh}{\mathbf{h}}
\newcommand{\vv}{\mathbf{v}}
\newcommand{\vx}{\mathbf{x}}
\newcommand{\vu}{\mathbf{u}}
\newcommand{\vy}{\mathbf{y}}
\newcommand{\vw}{\mathbf{w}}
\newcommand{\vs}{\mathbf{s}}
% 
\newcommand{\xf}{\mathbf{F}}
\newcommand{\xg}{\mathbf{G}}
\newcommand{\xh}{\mathbf{H}}
\newcommand{\xk}{\mathbf{K}}
\newcommand{\xl}{\mathbf{L}}
\newcommand{\xr}{\mathbf{R}}
%
\newcommand{\xu}{\mathbf{U}}
\newcommand{\xv}{\mathbf{V}}
\newcommand{\xx}{\mathbf{X}}
\newcommand{\xw}{\mathbf{W}}
\newcommand{\xy}{\mathbf{Y}}
\newcommand{\xz}{\mathbf{Z}}
\newcommand{\xa}{\mathbf{A}}
\newcommand{\xd}{\mathbf{D}}
% 
% with tildes
%% vectors
\newcommand{\vat}{\tilde{\vb}}
\newcommand{\vbt}{\tilde{\vb}}
\newcommand{\vct}{\tilde{\vc}}
\newcommand{\vht}{\tilde{\vh}}
\newcommand{\vvt}{\tilde{\vv}}
\newcommand{\vst}{\tilde{\vs}}
\newcommand{\vut}{\tilde{\vu}}
\newcommand{\vft}{\tilde{\vf}}
\newcommand{\xut}{\tilde{\xu}}
\newcommand{\vxt}{\tilde{\vx}}
\newcommand{\xvt}{\tilde{\xv}}
\newcommand{\xyt}{\tilde{\xy}}
%% matrices
\newcommand{\xwt}{\tilde{\xw}}
%
% with hats
\newcommand{\vhh}{\hat{\vh}}
\newcommand{\xvh}{\hat{\xv}}
\newcommand{\vvh}{\hat{\vv}}
\newcommand{\vyh}{\hat{\vy}}
\newcommand{\vxh}{\hat{\vx}}
\newcommand{\vuh}{\hat{\vu}}
\newcommand{\vfh}{\hat{\vf}}
\newcommand{\xyh}{\hat{\xy}}
\newcommand{\xxh}{\hat{\xx}}
\newcommand{\xuh}{\hat{\xu}}
%
%
% derivatives
\newcommand{\DRV}[2]{\frac{d #1}{d #2}}
\newcommand{\DRC}[3]{\DRV{#1}{#2}\DRV{#2}{#3}}
\newcommand{\PDV}[2]{\frac{\partial #1}{\partial #2}}
\newcommand{\PDC}[3]{\PDV{#1}{#2}\PDV{#2}{#3}}
%
% the diagnal matrix
\newcommand{\id}{\textrm{\textbf{I}}}
\newcommand{\im}{\textrm{\textbf{I}}}
% the vector of ones
\newcommand{\one}{\mathbf{1}}
% 
% xiaoran's edit
\newcommand{\xadd}[1]{\textcolor{blue}{#1}}
\newcommand{\xdel}[1]{\textcolor{red}{\sout{#1}}}
\newcommand{\xrpl}[2]{\xdel{#1}\xadd{#2}}
\newcommand{\xacc}[1]{\textcolor{ForestGreen}{#1}}
%
%
% declarations
% argument of the minimum / maximum
\DeclareMathOperator*{\argmin}{arg\,min}
\DeclareMathOperator*{\argmax}{arg\,max}
%
% norms
\newcommand\norm[1]{\left\lVert#1\right\rVert}

% \author{Xiaoran Tong}
\doublespacing
\begin{document}
\title{Notes on Stochastic Gradient}
\maketitle
\begin{flushleft}
  Xiaoran Tong\textsuperscript{1},
  \\
  \bigskip \textbf{1} Department of Epidemiology and Biostatistics,
  Michigan State University, East Lansing, USA

  \vskip 50ex
  Correspondence: Qing Lu\\
  Department of Epidemiology and Biostatistics\\
  College of Human Medicine\\
  Michigan State University\\
  909 Fee Road\\
  East Lansing, MI 48824-1030\\
  qlu@msu.edu\\
\end{flushleft}

\newcommand{\mvni}{\mathcal{N}(\bs{0}, \mathbf{I})}
\newcommand{\mvnp}[3]{\mathcal{N}(\mathbf{#1}, \mathbf{#2}; #3)}
\newcommand{\mvns}[2]{\mathcal{N}(\bs{#1}, \mathbf{#2})}
\newcommand{\funp}{\mean{f(\bs{\xi})}{\mvnp{\mu}{\Sigma}{\theta}}}
\newcommand{\funs}{\mean{f(\bs{\xi})}{\mvns{\mu}{\Sigma}}}
\newcommand{\dstp}{\mvnp{\mu}{\Sigma}{\theta}}
\newcommand{\dsts}{\mvns{\mu}{\Sigma}}

\section{Gradient of Mean}

\subsection{Multivariant Normal, MVN}
We are interested in calculating the gradient of an expectation form:
$\mean{f(\bs{\xi})}{\dstp}$ with respect to some paramter $\theta$,
that is,
\[ \PDV{\mean{f(\bs{\xi})}{\dstp}}{\theta}, \] where $\bs{\xi}$ is
drawn from the multivariate normal distribution of mean $\bs{\mu}$ and
covariance $\bs{\Sigma}$, parameterized by the very $\theta$ of
interests; $f$ is usually a loss function that we are interested in
minimizing via tuning $\theta$.

We first write down the gradient of $\funs$ with respect to the mean
$\bs{\mu}$,
\begin{align}\label{eq:gm} % gradient wrt. to mean
  \PDV{\funs}{\mu_i} = \mean{\PDV{f(\bs{\xi})}{\xi_i}}{\dsts},
\end{align}
and with respect to the covariance $\bs{\Sigma}$:
\begin{align}\label{eq:gc} % gradient wrt. to cov
  \PDV{\funs}{\sigma_{i, j}} =
  \frac{1}{2}\mean{\PDV{^2f(\bs{\xi})}{\xi_i\xi_j}}{\dsts},
\end{align}
\scriptsize Bonnet (1964) ``Transformations des signaux al´eatoires a
travers les syst`emes non lin´eaires sans m´emoire.'', \\
Price(1958) ``A useful theorem for nonlinear devices having Gaussian
inputs''.  \normalsize\\
These gradients are also of expectation form which requires a smaller
sample from $\dstp$ to estimate. Since the mean $\bs{\mu}$ and
covariance $\bs{\Sigma}$ are determined by $\theta$, the complete
gradient of $\funp$ w.r.t. $\theta$ is derived by applying the chain
rule to both (\ref{eq:gm}) and (\ref{eq:gc}), followed by additive
rule,
\begin{align}\label{eq:gf}
  \PDV{\mean{f(\bs{\xi})}{\dstp}}{\theta} =
  \mean{
  \vg^T \PDV{\bs{\mu}}{\theta} + 
  \frac{1}{2} \trb{\xh \PDV{\bs{\Sigma}}{\theta}}
  } {\dstp},
\end{align}
Here $\vg$ and $\xh$ are gradient and Hessian of $f(\bs{\xi})$, such
that the $i$ th entry of $\vg$, and the $(i, j)$ th entry of $\xh$ are
calculated by (\ref{eq:gm}) and (\ref{eq:gc}), respectively.

The trace term in (\ref{eq:gf}) can be seen as the chain of
derivatives passing through each each entry of covariance
$\bs{\Sigma}$, then being summed up since they all lead to the same
target parameter $\theta$,
\begin{align*}
  \sum_{i,j} \PDV{\mean{f(\bs{\xi})}{\dstp}}{\sigma_{i,j}}               \PDV{\sigma_{i,j}}{\theta} 
  &= \sum_{i,j} \frac{1}{2}\mean{\PDV{^2f(\bs{\xi})}{\xi_i\xi_j}}{\dstp} \PDV{\sigma_{i,j}}{\theta} \qquad (\textrm{use } \ref{eq:gc})\\
  &= \sum_{i,j} \frac{1}{2}\mean{\PDV{^2f(\bs{\xi})}{\xi_i\xi_j} \PDV{\sigma_{i,j}}{\theta}}{\dstp} \\
  &= \sum_{i,j} \frac{1}{2}\mean{\left[\xh\right]_{i,j} \left[\PDV{\bs{\Sigma}}{\theta}\right]_{i,j}}{\dstp} \\
  &= \mean{\frac{1}{2}\trb{\xh \PDV{\bs{\Sigma}}{\theta}}}{\dstp}
\end{align*}

\subsection{Multivariant Distribution in General (MVG)}

\subsection{Coordinate Transformations}
The stochastic gradient rules can be derived from transformation of a
standard distribution. In case of a random variable $\bs{\xi}$ from
$\mvns{\mu}{\Sigma}$, it can be alternatively generated from
$\bs{\xi} = \bs{\mu} + \xr\bs{\epsilon}$ where
$\bs{\epsilon} \sim \mvni$, and $\bs{\Sigma} = \xr\xr^T$, and the
paramter $\theta$ that determine the shape of $\bs{\Sigma}$ now
paramterize $\xr$ instead.

Part of the gradient that went throught the entries of covariance
$\bs{\Sigma}$ can now go through $\xr$ instead, that is,
\begin{align}\label{eq:gr} % gradient wrt. to decomposed cov
  \PDV{\funs}{\xr} =
  \PDV{\mean{f(\bs{\mu} + \xr \bs{\epsilon})}{\mvni}}{\xr} =
  \mean{\bs{\epsilon}\vg^T}{\mvni},
\end{align}
and
\begin{align*}
  \sum_{i,j} \PDV{\mean{f(\bs{\xi})}{\dstp}}{\sigma_{i,j}} \PDV{\sigma_{i,j}}{\theta}
  = & \sum_{i,j}       \PDV{f(\bs{\mu} + \xr\bs{\epsilon})}{r_{i,j}}                \PDV{r_{i,j}}{\theta} \\
  = & \sum_{i,j} \left[\PDV{f(\bs{\mu} + \xr\bs{\epsilon})}{\xr}\right]_{i,j} \left[\PDV{\xr}{\theta}\right]_{i,j} \\
  = & \sum_{i,j} \left[\mean{\bs{\epsilon}\vg^T}{\mvni}         \right]_{i,j} \left[\PDV{\xr}{\theta}\right]_{i,j} \\
  = & \sum_{i,j} \mean{\left[\bs{\epsilon}\vg^T\right]_{i,j} \left[\PDV{\xr}{\theta}\right]_{i,j}}{\mvni} =
      \mean{\trb{\left[\bs{\epsilon}\vg^T\right]\left[\PDV{\xr}{\theta}\right]}}{\mvni},
\end{align*}
where $\vg$ is the gradient of $f(\bs{\xi})$ wrt
$\bs{\xi} = \bs{\mu} + \xr\bs{\epsilon}$, which gets a new value for
each sample of $\bs{\epsilon} \sim \mvni$; as an reminder, $vr$
replaces $\bs{\Sigma}$ and is now paramterized by $\theta$.

The complete gradient rule now become
\begin{align}
  \PDV{\mean{f(\bs{\xi})}{\dstp}}{\theta} =
  \mean{
  \vg^T \PDV{\bs{\mu}}{\theta} + 
  \trb{\left[\bs{\epsilon}\vg^T\right]\left[\PDV{\xr}{\theta}\right]}
  } {\mvni},
\end{align}
which is a simplified version of Price's (\ref{eq:gc}).
% Hessian of $f(\bs{\xi})$ and the

\section{Stochastic Gradient for Deep Network}
To compute the gradient of some objective function over the network's output,
with respect to the flexible paramters of that network, all its latent data
$\bs{\xi}$ has to be integrated out first. Such a task can be both analytically
and numerically intractable.

Instead we seek to maximum one of the lower bounds of the objective function,
or equivalently, minimize the upper bound of a loss function.

The generic loss function of a generative deep net over an observed reality $\xv$
is,
\begin{align*}
  \mathcal{L}(\xv) = -\log p(\xv) = -\log \int p(\xv|\bs{\xi}, \theta) p(\bs{\xi}, \theta) d\,\bs{\xi}
\end{align*}

\singlespacing \bibliographystyle{\style} \bibliography{ref}

\end{document}
